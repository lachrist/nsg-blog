\documentclass{article}
\usepackage[utf8]{inputenc}
\usepackage{fullpage}
\usepackage{url}
\usepackage{txfonts}

\begin{document}
\pagenumbering{gobble}
\begin{center}\huge{Meeting International Richard Anis}\end{center}

Samedi 21 novembre et dimanche 22 novembre 2015, les nageurs du groupe pré-compet (Miguel) et compet (Laurent) particeperont au meeting internal Richard Anis.
Le meeting se déroulera à Louvain-la-neuve et l'avant programme peut être téléchargé sur \texttt{\url{http://www.ffbn.be/}} en cliquant dans le calendrier sur le jour du 21 novembre.
Comme la competition commence à 08h00 dimanche, nous proposons à ceux qui nagent samedi de rester loger sur place à Louvain-la-neuve.
Bien sûr ceci n'est pas obligatoire et vous êtes toujours libre de vous organiser par vos propre moyens, l'important est que vous soyez à temps à la competition!
En cas de problème vous pouvez me contacter par téléphone au 0472/816/228.

\paragraph{Info competition:}
\begin{itemize}
\item \textbf{Où:}
Piscine du Blocry: Rue de Castinia, 1348 Louvain-la-Neuve.
\item \textbf{Quand:}
Samedi 21 novembre de 16h30 à environ 20h30 et dimanche 22 novembre de 8h00 à environ 20h30.
Attention, les canetons -- c'est à dire: Daniel et Luis -- ne nagent que dimanche.
\item \textbf{Comment:}
Un départ groupé est prévu à la gare du midi samedi 21 novembre à 14h45, \textbf{ne soyez pas en retard sinon on rate le train!}
De même, nous avons prévu de rentrer en train après la compétition.
Nous serons de retour à la gare du midi dimanche soir vers 21h00.
\end{itemize}

\paragraph{Info logement:}
\begin{itemize}
\item \textbf{Où:}
Centre Adeps du Blocry: Place des Sports 3, 1348 Louvain-La-Neuve.
Plus d'information disponible sur leur site internet: \texttt{\url{http://www.sport-adeps.be/index.php?id=cs_lln}}
\item \textbf{Quand:}
Le check-in est prévu samedi à 16h00 et le check-out est prévu dimanche à 7h15.
\item \textbf{Prix:}
Le prix du logement, petit déjeuner inclus, s'élève à 20 euros par personne.
Comme d'habitude, nous pouvons toujours nous arranger si le prix pose problème.
Pour sa part, le club prend en charge l'aller-retour en train et les droits d'inscriptions à la compétition.
\end{itemize}

\paragraph{A prendre:}
\begin{enumerate}
\item Maillot de bain (duhhh).
\item Lunettes de natation qui tiennent au plongeon.
\item Bonnet en \textbf{silicone} (pas en tissus), du club de préférence.
\item Plusieurs essuits et de quoi ce tenir chaud entre les courses comme par exemple un pull et un training.
\item De quoi boire, prévoir environ 6 litres pour les deux jours!
Bien s'hydrater est important, surtout pour un longue compétition tel que celle-ci.
Boisson type: eau, aquarius, isostar, grenadine.
\textbf{Pas} de sodas trop sucré comme du coca et du fanta ni de redbull et autre boisson ``énergétique'' à la con.
\item De quoi manger: repas de dimanche soir et samedi midi plus des collations pour grignoter entre les courses.
Example de collation: banane, fruits secs et biscuits aux céréals comme par example: grani, petit beurre et dinausaures.
\textbf{Pas} de bonbons ni de barres au chocolat tel que snickers et lion.
La cafetaria du blocry étant ouverte tout le weekend, les nageurs pourront toujours acheter un repas equilibré sur place.
\end{enumerate}
\vspace{1cm}
Au recto, veuillez trouver la liste des courses par nageur.\\
Sportivement votre,\\
Laurent

{\scriptsize\begin{tabular}{|p{5cm}|p{5cm}|p{5cm}|}
\hline
\begin{verbatim}
====================
== Groupe Laurent ==
====================

Fiona
  == samedi soir ==
  02: 400 nl
  08: 50nl
  == dimanche matin ==
  14: 100 nl
  22: 50 pap
  == dimanche aprem ==
  28: 200 nl
  36: 100 4n

Mohamed
  == samedi soir ==
  01: 100 dos
  07: 50 pap
  == dimanche matin ==
  15: 200 nl
  23: 100 pap
  == dimanche aprem ==
  29: 100 nl
  35: 200 4n

Bilal
  == samedi soir ==
  05: 50 brasse
  09: 200 brasse
  == dimanche matin ==
  19: 100 brasse
  25: 100 4n
  == dimanche aprem ==
  29: 100 nl
  37: 50 nl

Matthias VDB
  == samedi soir ==
  07: 50 pap
  11: 400 nl
  == dimanche matin ==
  15: 200 nl
  23: 100 pap
  == dimanche aprem ==
  29: 100 nl
  35: 200 4n

Mathias LM 
  == samedi soir ==
  05: 50 brasse
  07: 50 pap
  == dimanche matin ==
  15: 200 nl
  19: 100 brasse
  == dimanche aprem ==
  29: 100 nl
  37: 50 nl

Nassif
  == samedi soir ==
  07: 50 pap
  == dimanche matin ==
  17: 50 dos
  == dimanche aprem ==
  37: 50 nl

Antonios
  == samedi soir ==
  05: 50 brasse
  == dimanche matin ==
  19: 100 brasse
  25: 100 4n
  == dimanche matin ==
  29: 100 nl
  37: 50 nl

Aline
  == samedi soir ==
  02: 400 nl
  06: 50 nl
  == dimanche matin ==
  14: 100 nl
  22: 50 pap
  == dimanche aprem ==
  28: 200 nl
  36: 100 4n
\end{verbatim}
&
\begin{verbatim}
===================
== Groupe Miguel ==
===================

Daniel
  21: 100 4n (caneton)
  27: 100 brasse (caneton)
  34: 100 nl (caneton)

Luis
  21: 100 4n (caneton)
  27: 100 brasse (caneton)
  34: 100 nl (caneton)

Evmorfia
  10: 100 brasse
  14: 100 nl
  28: 200 nl

Regina
  10: 100 brasse
  14: 100 nl
  28: 200 nl

Jana
  10: 100 brasse
  14: 100 nl
  28: 200 nl

Habib
  01: 100 dos
  15: 200 nl
  29: 100 nl

Raed
  01: 100 dos
  15: 200 nl
  29: 100 nl

Felix
  01: 100 dos
  15: 200 nl
  29: 100 nl

Patrick
  01: 100 dos
  15: 200 nl
  29: 100 nl

Tiago
  01: 100 dos
  15: 200 nl
  29: 100 nl
\end{verbatim}
&
\begin{verbatim}
============
== Relais ==
============

1 x (12: 4x100 4n Dames)
2 x (13: 4x100 4n Messieurs)

1 x (39: 4x100 nl Dames)
2 x (40: 4x100 nl Messieurs)

2 x (41: 4x50 nl Mixte)
\end{verbatim}\\
\hline
\end{tabular}}

\end{document}