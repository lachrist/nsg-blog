\documentclass{article}
\usepackage[utf8]{inputenc}
\usepackage{fullpage}
\usepackage[parfill]{parskip}

\begin{document}

\begin{center}\huge{Compétition Samedi 24 Mai}\end{center}
\hspace{1cm}

Chers nageurs, chers parents,

Ce samedi 24 mai, nous participons à la compétition du CNSW.
Départs groupés à la piscine de Saint-Gilles pour 11h30, ne soyez pas en retard.

\vspace{0.5cm}

\underline{Adresse:}\\
Piscine du Sporcity Woluwé\\
Avenue Salomé, 2\\
1150 Woluwe Saint Pierre\\

\underline{Programme du jour:}
\begin{itemize}
\item 13h00 ouverture des portes et échauffement
\item 14h00 début des courses
\item Courses:
\begin{itemize}
\item 200 Nage libre messieurs
\item 50 brasse dames
\item 50 brasse messieurs
\item 50 nage libre dames
\item 50 nage libre messieurs
\item 200 papillon dames
\item 200 papillon messieurs
\item 200 dos dames
\item 200 dos messieurs
\item 400 4 nages dames
\item 400 4 nages messieurs
\item Pause 20 minute
\item 400 nage libre dames
\item 400 nage libre messieurs
\item 100 brasse dames
\item 100 brasse messieurs
\item 100 nage libre dames
\item 100 nage libre messieurs
\end{itemize}
\item Fin estimée vers 19h00
\end{itemize}

\vspace{1cm}

Vous devez vous munir du bonnet du club, du tshirt ainsi que du sac du club!
Noubliez pas de prendre des bonnes lunettes qui tiennent au plongeon.
Ensuite, nous vous suggérons de prendre plusieurs essuies pour vous sécher, des clapettes/ tongues pour ne pas vous refroidir, ainsi qu’un short qui peut être mouillé. Si vous avez plusieurs maillots, ce ne sera pas un luxe.
Question nourriture, prenez des aliments qui se digèrent facilement (un tupperware de pâtes est toujours facile et bon en cas d’effort) sans oublier de l’eau avec de la grenadine ou quelque chose d’un peu sucré.

Bons préparatifs et bon courage à tous!

Les entraîneurs

\end{document}
