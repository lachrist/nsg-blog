\documentclass{article}
\usepackage[utf8]{inputenc}
\usepackage{fullpage}
\usepackage[parfill]{parskip}

\begin{document}

\begin{center}\huge{Compétition samedi 5 octobre}\end{center}
\hspace{1cm}

Chers nageurs, chers parents,

Ce samedi 5 octobre, nous participons à la compétition ``\emph{Journées Nautiques}'' à Gosselies.
Départs groupés possibles pour au cas ou les parents ne peuvent pas conduire leurs enfants.

Adresse:\\
Piscine Aqua 2000\\
Chaussée de Nivelles, 37\\
6041 Gosselies\\

Programme du jour:
\begin{itemize}
\item 8h ouverture des portes et échauffement
\item 9h début des courses
\item Pause midi de 12 à 13h
\item Echauffement à 13h
\item Reprise des courses à 14h
\item Fin estimée vers 17h environ
\end{itemize}
\begin{table}[here!]
\centering
\begin{tabular}{|l|l|l|l|}
\hline
Alkai, Titsiana & 400 4n & 200 dos & 200 nl\\
\hline
Francis, Emma & 400 nl & 100 dos &\\
\hline
Pestieau, Fiona & 400 nl & 100 dos &\\
\hline
Ucar, Anaïs & 200 nl & 200 dos &\\
\hline
Alkai, Kevin & 400 nl & 200 4n & 200 brasse\\
\hline
Almachi-Mendoza, Angelo & 400 nl & 200 4n &\\
\hline
Delobelle, Maxime & 400 nl & 200 brasse & 200 4n\\
\hline
Francis, Jérémis & 400 nl & 200 4n & 200 brasse\\
\hline
Kordoghli, Ghassen & 400 nl & 200 4n & 200 brasse\\
\hline
Le Maire de Romsée, Mathias & 400 nl & 200 4n & 200 brasse\\
\hline
\end{tabular}
\end{table}

Vous devez vous munir du bonnet du club, du tshirt ainsi que du sac du club!
Noubliez pas de prendre des bonnes lunettes qui tiennent au plongeon.
Ensuite, nous vous suggérons de prendre plusieurs essuies pour vous sécher, des clapettes/ tongues pour ne pas vous refroidir, ainsi qu’un short qui peut être mouillé. SI vous avez plusieurs maillots, ce ne sera pas un luxe.
Question nourriture, prenez des aliments qui se digèrent facilement (un tupperware de pâtes est toujours facile et bon en cas d’effort) sans oublier de l’eau avec de la grenadine ou quelque chose d’un peu sucré.

Bons préparatifs et bon courage à tous!

Les entraîneurs

\end{document}

Dimanche 28 avril, nous avons prévu une compétition pour le groupe de Laurent.
Il s'agit d'une compétition non licensiée pour les nageurs débutants et relativement jeunes (seuls les 12 ans et moins seront repris dans le classement pour les médailles).
Le but étant de familiariser le nageur à l'univers de la compétition en douceur.

\paragraph{Info pratique:}
\begin{itemize}
\item \textbf{Ou:} Piscine Calypso, 60 avenue L. Wiener 1170 Watermael-Boitsfort
\item \textbf{Quand:} Dimanche 28 avril à 13h30
\item \textbf{Comment:} Pour ceux qui ne peuvent pas se faire conduire, un départ groupé est prévu à la piscine de Saint-Gilles à 12h15. \textbf{Ne soyez pas en retard!}
\end{itemize}

\paragraph{A prendre:}
\begin{enumerate}
\item Maillot de bain
\item Lunette de natation
\item Bonnet en \textbf{silicone} (pas en tissus)
\item De quoi boire ($\pm$ 1.5l): aquarius, eau, grenadine
\item De quoi manger: banane, fruits secs, biscuit aux cereals (e.g. petit beurre)
\end{enumerate}

\begin{table}[here!]
\centering
\begin{tabular}{|l|l|}
\hline
13h30 & Ouverture des portes\\
\hline
14h00 & Premier départ\\
\hline
Course 1 & 200 m crawl filles\\
\hline
Course 2 & 200 m crawl garçons\\
\hline
Course 3 & 100 m brasse filles\\
\hline
Course 4 & 100 m brasse garçons\\
\hline
\multicolumn{2}{|l|}{Pause \& remise des médailles des courses 1 à 4}\\
\hline
Course 5 & 100 m dos filles\\
\hline
Course 6 & 100 m dos garçons\\
\hline
Course 7 & 100 m crawl filles\\
\hline
Course 8 & 100 m crawl garçons\\
\hline
\multicolumn{2}{|l|}{Remise des médailles des courses 4 à 8}\\
\hline
\end{tabular}
\caption{Program de la compétition}
\end{table}

\end{document}