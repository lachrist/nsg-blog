\documentclass{article}
\usepackage[utf8]{inputenc}
\usepackage{fullpage}
\usepackage[parfill]{parskip}

\begin{document}

\begin{center}\huge{Compétition Dimanche 22 Décembre}\end{center}
\hspace{1cm}

Chers nageurs, chers parents,

Ce samedi 22 mars, nous participons à la compétition du CNSW.
Départs groupés à la piscine de Saint-Gilles pour 11h30, ne soyez pas en retard.

\vspace{0.5cm}

\underline{Adresse:}\\
Piscine du Sporcity Woluwé\\
Avenue Salomé, 2\\
1150 Woluwe Saint Pierre\\

\underline{Programme du jour:}
\begin{itemize}
\item 16h45 ouverture des portes et échauffement
\item 17h45 début des courses
\item Courses:
\begin{itemize}
\item 100 Papillon messieurs
\item 100 Papillon dames
\item Pause 10 minutes
\item 100 Nage libre canetons garçons
\item 100 Nage libre canetons filles
\item 100 Dos messieurs
\item 100 Dos dames
\item 100 Brasse messieurs
\item 100 Brasse dames
\item Pause 10 minutes
\item 100 Nage libre canetons messieurs
\item 100 Nage libre canetons dames
\end{itemize}
\item Fin estimée vers 21h30
\end{itemize}

\vspace{1cm}

Vous devez vous munir du bonnet du club, du tshirt ainsi que du sac du club!
Noubliez pas de prendre des bonnes lunettes qui tiennent au plongeon.
Ensuite, nous vous suggérons de prendre plusieurs essuies pour vous sécher, des clapettes/ tongues pour ne pas vous refroidir, ainsi qu’un short qui peut être mouillé. Si vous avez plusieurs maillots, ce ne sera pas un luxe.
Question nourriture, prenez des aliments qui se digèrent facilement (un tupperware de pâtes est toujours facile et bon en cas d’effort) sans oublier de l’eau avec de la grenadine ou quelque chose d’un peu sucré.

Bons préparatifs et bon courage à tous!

Les entraîneurs

\end{document}
